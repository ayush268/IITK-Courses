\documentclass{article}
\usepackage[a4paper]{geometry}
\usepackage[utf8]{inputenc}
\usepackage[english]{babel}
\usepackage{lipsum}
\usepackage{changepage}

\usepackage{amsmath, amssymb, amsfonts, amsthm, fouriernc, mathtools}
\usepackage{microtype}

\usepackage[svgnames]{xcolor}
\definecolor{lightgrey}{rgb}{0.5,0.5,0.5}
\definecolor{grey}{rgb}{0.25,0.25,0.25}
\newcommand{\blackb}{\color{Black} \usefont{OT1}{lmss}{m}{n}}
\newcommand{\lightgreyb}{\color{lightgrey} \usefont{OT1}{lmss}{m}{n}}

\let\bold\textbf
\newcommand\comb[2][^n]{\prescript{#1\mkern-0.5mu}{}C_{#2}}

\usepackage{titlesec}
\usepackage{sectsty}
\sectionfont{\color{lightgrey}}
\subsectionfont{\color{lightgrey}}
\subsubsectionfont{\color{lightgrey}}

\renewcommand\thesection{\Roman{section}}
\renewcommand\thesubsection{\arabic{section}.\arabic{subsection}}
\renewcommand\thesubsubsection{\arabic{section}.\arabic{subsection}.\arabic{subsubsection}}

\usepackage{chngcntr}
\counterwithin*{equation}{section}

\newcommand{\mysection}{
\titleformat{\section} [runin] {\usefont{OT1}{lmss}{b}{n}\color{lightgrey}}
{\thesection} {3pt} {} }

\renewcommand{\theequation}{\arabic{section}.\arabic{equation}}

\usepackage{etoolbox}
\makeatletter
\patchcmd{\@Aboxed}{\boxed{#1#2}}{\colorbox{black!15}{$#1#2$}}{}{}
\patchcmd{\@boxed}{\boxed{#1#2}}{\colorbox{black!15}{$#1#2$}}{}{}
\makeatother

\title{\vspace{80mm}\lightgreyb CS201: Math for CS I/Discreet Mathematics \\
\lightgreyb Assignment $3$ Solutions}
\author{Ayush Bansal \\
Roll No. 160177}
\date{\today}

\newtheorem{theorem}{Theorem}
\newtheorem{corollary}{Corollary}[theorem]
\newtheorem{conjecture}{Conjecture}
\newtheorem{lemma}{Lemma}[section]
\newtheorem{claim}{Claim}[subsection]
\newenvironment{solution}
  {\begin{proof}[Solution]}
  {\end{proof}}
\AfterEndEnvironment{theorem}{\noindent\ignorespaces}
\renewcommand\thelemma{\arabic{section}.\arabic{lemma}}

\newenvironment{myenv}{\begin{adjustwidth}{1cm}{}}{\end{adjustwidth}}

\begin{document}
\clearpage\maketitle
\thispagestyle{empty}
\newpage
\setcounter{page}{1}
\section{Problem 1 Solution}{
  \subsection{Part (a)}{
    We are given a \bold{simple} graph $G=(10,28)$.
    \begin{claim}
      $G$ has a cycle of length 4.
    \end{claim}
    \begin{proof}
      Since the number of vertices and edges in the graph are $10$ and $28$ respectively, the average degree $d(G)=\frac{2\mid E\mid}{\mid V\mid}$ of the graph is $5.6$ and thus, we can observe that there has to be atleast $2$ vertices with degree sum atleast $12$ i.e. $deg(v)+deg(u) \geq 12$. \newline
      If $u$ and $v$ are not adjacent to each other then, they are connected to $12$ other vertices but there are only $8$ vertices left so there will be atleast $4$ common vertices between the $2$ which gives us a cycle of $4$. \newline
      If $u$ and $v$ are adjacent to each other then their degree sum is lowered to $10$ and they are connected to $10$ other vertices but again there are only $8$ vertices left so there will be atleast $2$ common vertices between the $2$ which again gives us a cycle of $4$
    \end{proof}
  }
  \subsection{Part (b)}{
    We are given a \bold{simple} graph $G=(10,38)$. I first make the following claim:
    \begin{claim}
      There are atleast $3$ vertices such that $deg(v) \geq 8$.
    \end{claim}
    \begin{proof}
      Since the graph is simple, the maximum degree a vertex can have is $\mid V \mid -1$ i.e. $9$ in this case, also the sum of degrees of all vertices will be $2 \mid E \mid = 76$. \newline
      Assume $8$ vertices have $deg(v) \leq 7$ which means the sum of their degrees is $\leq 56$ but that would mean the sum of degrees of other $2$ vertices is $\geq 20$ which is not possible, but now consider $7$ vertices have $deg(v) \leq 7$ which means the sum of their degrees is $\leq 49$ and this is possible if other $3$ vertices have $deg(v)=9$. \newline
      The above arguments prove that there are atleast $3$ vertices such that $deg(v) \geq 8$.
    \end{proof}
    Now I will use the result of the above claim to prove the following claim.
    \begin{claim}
      $G$ contains $K_4$ as an induced sub-graph.
    \end{claim}
    \begin{proof}
      Consider the $3$ vertices $v_1,v_2,v_3$ such that their degree is $8$, now since $v_1$ has degree $8$, it will connected to $8$ vertices out of which atleast $1$ will be one out of $v_2,v_3$, same goes for $v_2$ and $v_3$, so $1$ vertex out of $3$ is connected to other $2$. \newline
      So finally we have edges $(v_1,v_2),(v_2,v_3)$ and there will a vertex $v_4$ such that it is connected to all $3$ earlier vertices, now we have $2$ cases: \newline
      \bold{Case 1.} There is an edge between $v_1,v_3$
      \begin{myenv}
        If there is an edge $(v_1,v_3)$, then we are done and our \bold{induced subgraph $K_4$} is $v_1,v_2,v_3,v_4$.
      \end{myenv}
      \bold{Case 2.} There is no edge between $v_1,v_3$
      \begin{myenv}
        Since the remaining degree of $v_1$ is $6$, it will be connected to every other node of the graph and since we know that degree of any node can't be $ \leq 2$, there will be a node $v_5$ such that it is connected to $v_4,v_1,v_2$ which gives us an \bold{induced graph $K_4$} i.e. graph of nodes $v_1,v_2,v_4,v_5$.
      \end{myenv}
    \end{proof}
  }
  \subsection{Part (c)}{
    We will find the number of \bold{automorphisms} for each graph in the following cases:
    \subsubsection{Part i}{
      We are given a \bold{complete} graph $K_n$.
      \begin{solution}
        Every node of this graph has degree $n-1$ and it is connected to every other node of the graph. \newline
        Since all nodes have degree $n-1$, thus we can consider each node as equivalent while finding the number of automorphisms, thus the final number will be:
        \begin{align*}
          \comb[n]{1}(n-1)!=n!
        \end{align*}
        Thus, the final answer will be $n!$.
      \end{solution}
    }
    \subsubsection{Part ii}{
      We are given a \bold{Cycle} graph $C_n$.
      \begin{solution}
        Every node of this graph has degree $2$ and thus degree of every node afterwards must also remain the same. \newline
        Since all nodes have degree $2$, thus we can consider each node as equivalent while finding the number of automorphisms, thus the final answer will be:
        \begin{align*}
          \comb[n]{1}2!=2n
        \end{align*}
        Thus, the final answer will be $2n$.
      \end{solution}
    }
    \subsubsection{Part iii}{
      We are given a \bold{Path} graph $P_n$.
      \begin{solution}
        The graph is considered to be a straight line with $n$ nodes i.e. $2$ nodes with degree $1$ and other nodes with degree $2$. \newline
        Thus the possible \bold{number of automorphisms} of this graph is $\bold{2}$ as any other automorphism will either changes the degree of the vertices or changes the relative order of the vertices.
      \end{solution}
    }
    \subsubsection{Part iv}{
      We are given a \bold{Star} graph $S_n$.
      \begin{solution}
        In this graph there is $1$ nodes of degree $n-1$ and all other nodes of degree $1$ which means the node with degree $n-1$ lies at the centre and all other nodes are connected to it. \newline
        Now counting the possible number of automorphisms, the node at the centre will remain the same and other nodes can be considered equivalent which means the \bold{number of automorphisms} is $\bold{(n-1)!}$.
      \end{solution}
    }
  }
  \subsection{Part (d)}{
    We are given a \bold{simple} graph $G$ such that $\delta(G)=k$ where $\delta(G)$ is the minimum degree of the graph.
    \begin{claim}
      $G$ has a cycle of length atleast $k+1$.
    \end{claim}
    \begin{proof}
      Since the degree of each vertex is $k$, there has to be atleast $k+1$ vertices in the graph since it is a simple graph. \newline
      Consider $x_0,\dots x_n$ to be a longest path in $G$, then all the neighbours of $x_n$ lie on this path, thus we can say that $n \geqslant d(x_n) \geqslant k$. \newline
      If $i<n$ is minimal with $x_ix_n \in E(G)$, then $x_i\dots x_nx_i$ is a cycle of length atleast $k+1$.
    \end{proof}
  }
}
\section{Problem 2 Solution}{
  \subsection{Part (a)}{
    We are given a \bold{simple} graph $G=(V,E)$ with $n$ nodes such that for distinct $u,v \in V$, we have $deg(u)+deg(v)\geq n$.
    \subsubsection{Part i}{
      I have to prove the following claim
      \begin{claim}
        $G$ is connected.
      \end{claim}
      \begin{proof}
        I will prove the claim by \bold{contradiction}. \newline
        Assume the graph $G$ is disconnected. \newline
        Since the graph is disconnected, there will be atleast $2$ connected different sub-graphs of the graph which have no edge between them, let these components be $A$ and $B$. \newline
        $\therefore$ there will be non-adjacent vertices $u$ and $v$ in $A$ and $B$ respectively, let $A$ and $B$ have total number of $a$ and $b$ vertices, so $a+b \leq n$. \newline
        The graph is simple thus a node in $A$ can have a max degree of $\mid A\mid -1$ which means $deg(u) \leq a-1$ and $deg(v) \leq b-1$.
        \begin{align*}
          deg(u)+deg(v) &\leq a+b-2 \\
          deg(u)+deg(v) &< a+b \\
          deg(u)+deg(v) &< n
        \end{align*}
        The above equation is clearly a \bold{contradiction} as $deg(u)+deg(v) \geq n$ and thus graph $G$ is connected.
      \end{proof}
    }
    \subsubsection{Part ii}{
      I have to prove the following claim.
      \begin{claim}
        $G$ has a $\bold{H-cycle}$.
      \end{claim}
      \begin{proof}
        I will prove the claim by \bold{contradiction}. \newline
        Assume the graph has no $\bold{H-cycle}$. \newline
        Since we know graph is connected as we proved above as \bold{Claim 2.1.1}, we consider a graph $G$ with the most possible edges such that the graph $G$ is not Hamiltonian. \newline
        Since the graph $G$ has the most possible edges without an $H-cycle$, adding $1$ edge must make it Hamiltonian, thus there must be a \bold{Hamiltonian Path} - $\{v_1,v_2\dots v_n\}$ in the graph. \newline
        Since we know, $G$ has no $H-cycle$, nodes $v_1$ and $v_n$ do not have an edge between them, now consider $v_i$ is neighbour of $v_1$, then $v_{i-1}$ can't be the neighbour of $v_n$, otherwise we will get a $H-cycle$ which is $\{v_1,v_2\dots,v_{i-1},v_n,v_{n-1},\dots,v_i,v_1\}$. \newline
        Since, if $v_i$ is neighbour of $v_1$ then $v_{i-1}$ is not the neighbour of $v_n$ which gives us 
        \begin{align}
          deg(v_1)+deg(v_n) \leq n-1 \label{eq:1}
        \end{align}
        Equation (2.1) is clearly a \bold{contradiction} as $deg(v_1)+deg(v_n) \geq n$.
      \end{proof}
    }
    \subsubsection{Part iii}{
      Now consider \bold{simple} graph $G=(V,E)$ such that $\mid V\mid =n$ and $\forall u,v \in V$ we have $deg(u)+deg(v) \geq n-1$.
      \begin{claim}
        \bold{Claim 2.1.2} does not hold for the above mentioned graph.
      \end{claim}
      \begin{proof}
      In this case it is possible that the graph has no $H-cycle$ as by equation (2.1) we can see that it is possible to have $deg(v_1)+deg(v_n)=n-1$ and not be a \bold{Hamiltonian} graph. \newline
      Taking an example graph $G=(4,4)$ such that $V=\{v_1,v_2,v_3,v_4\}$ and $E=\{(v_1,v_2),(v_1,v_3),(v_2,v_3),(v_3,v_4)\}$, this graph does not have a $H-cycle$ but satisfies the condition $deg(v_i)+deg(v_j) \geq 3$.
    \end{proof}
    }
  }
  \subsection{Part (b)}{
    We are given a \bold{simple} graph $G=(V,E)$ with $\mid V\mid =n$ and $\forall v \in V, deg(v)\geq \frac{n}{2}$.
    \begin{claim}
      $G$ has a \bold{H-cycle}.
    \end{claim}
    \begin{proof}
      I will prove the claim by \bold{contradiction}, so lets assume $G$ doesn't have a $H-cycle$. \newline
      Since $deg(v)\geq \frac{n}{2}$, consider $2$ nodes in $V$, $deg(u)+deg(v) \geq n$, thus by \bold{Claim 2.1.1}, graph $G$ is connected. \newline
      Now consider a path $P$ and let this path be the longest path of the graph $G$, $P=\{v_1,v_2\dots v_k\}$, and since it is the longest path $v_1$ and $v_k$ cannot have an edge with some other vertex $u$ which is not in path $P$ as it will yield a longer path. \newline
      Since $deg(v_1) \geq \frac{n}{2}$, and $v_1$ can have edges only with vertices in path $P$, we have $k \geq \frac{n}{2} +1$. \newline
      Consider a vertex $v_j$ in this path, if it is adjacent to $v_k$, then $v_{j+1}$ can't be adjacent to $v_1$ otherwise the path $P$ will form a cycle $\{v_1,v_{j+1},\dots,v_{k-1},v_k,v_j\dots,v_2,v_1\}$ and since $G$ is connected this path $P$ will cover all the nodes of the graph otherwise if there are other nodes, $G$ will become disconnected. \newline
      Since, if $v_j$ is adjacent to $v_k$, $v_{j+1}$ can't be adjacent to $v_1$ which gives us
      \begin{align*}
        k &\geq deg(v_1)+deg(v_k)+1 \\
        k &\geq n+1
      \end{align*}
      But the above equation is clearly a \bold{contradiction} as there can be only $n$ nodes.
    \end{proof}
  }
}
\newpage
\section{Problem 3 Solution}{
  \subsection{Part (a)}{
    We are given a complete graph $K_n$ and I have to find the number of different \bold{Hamiltonian Cycles} in the graph.
    \begin{solution}
      Since the graph is complete, every vertex is adjacent to every other vertex and thus we can take vertices in any order. \newline
      Since there are $n$ vertices, $n!$ ways are possible of ordering the vertices in a cycle and since for having different H-cycle, the edge set must be different, so we need to eliminate the similar cases. \newline 
      We need to divide the answer by $\comb[n]{1}$ since we can start from any vertex, also since a cycle is same backwards and forwards, we need to eliminate those cases, thus dividing by $2$, the final answer will be
      \begin{align*}
        \frac{n!}{2\comb[n]{1}}=\frac{(n-1)!}{2}
      \end{align*}
    \end{solution}
  }
  \subsection{Part (b)}{
    We are given a bi-partite complete graph $K_{m,n}$ where $\mid V_1 \mid =m$ and $\mid V_2 \mid =n$ and $mn$ edges connect nodes in $V_1$ to nodes in $V_2$.
    \begin{claim}
      $K_{m,n}$ has no \bold{H-cycle} when $m \neq n$.
    \end{claim}
    \begin{proof}
      Since the graph is complete bi-partite, there is an edge between each node of $V_1$ and $V_2$, so we can consider nodes of $V_1$ and $V_2$ equivalent among themselves. \newline
      For traversing between the $2$ sets of vertices, we will move alternately as the graph is bi-partite. \newline
      Since $m \neq n$, let $m>n$, assume that $G$ has a $H-cycle$, let cycle start at some node $u \in V_2$, so after traversing $2n$ edges, we will arrive back at $u$ and since $m>n$, there will $m-n$ vertices left in $V_1$ which will not be visited. \newline
      By above arguments, it is clear $K_{m,n}$ will not have a \bold{H-cycle} when $m \neq n$.
    \end{proof}
    Now, we consider $m=n$ and find the number of \bold{H-cycles}.
    \begin{solution}
      Consider a vertex $v \in V_1$ and let's start the cycle from this point, since all nodes are equivalent, we can begin from any vertex and finally reach at that. \newline
      Now begin from $v$, we can choose $1$ of $n$ vertices from $V_2$, then $1$ of $n-1$ vertices left of $V_1$ and continuing this we get $n!(n-1)!$ and since a cycle is same forwards and backwards, the final answer will be to divide it by $2$. \newline
      Thus the answer when $m=n$ is $\frac{n!(n-1)!}{2}$.
    \end{solution}
  }
}
\end{document}
