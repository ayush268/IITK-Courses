\documentclass{article}
\usepackage[a4paper]{geometry}
\usepackage[utf8]{inputenc}
\usepackage[english]{babel}
\usepackage{lipsum}
\usepackage{changepage}

\usepackage{amsmath, amssymb, amsfonts, amsthm, fouriernc, mathtools}
\usepackage{microtype}

\usepackage[svgnames]{xcolor}
\definecolor{lightgrey}{rgb}{0.5,0.5,0.5}
\definecolor{grey}{rgb}{0.25,0.25,0.25}
\newcommand{\blackb}{\color{Black} \usefont{OT1}{lmss}{m}{n}}
\newcommand{\lightgreyb}{\color{lightgrey} \usefont{OT1}{lmss}{m}{n}}

\usepackage{titlesec}
\usepackage{sectsty}
\sectionfont{\color{lightgrey}}
\subsectionfont{\color{lightgrey}}

\renewcommand\thesection{\arabic{section}.}
\renewcommand\thesubsection{\thesection\arabic{subsection}}

\usepackage{chngcntr}
\counterwithin*{equation}{section}

\newcommand{\mysection}{
\titleformat{\section} [runin] {\usefont{OT1}{lmss}{b}{n}\color{lightgrey}}
{\thesection} {3pt} {} }

\renewcommand{\theequation}{\thesection\arabic{equation}}

\usepackage{etoolbox}
\makeatletter
\patchcmd{\@Aboxed}{\boxed{#1#2}}{\colorbox{black!15}{$#1#2$}}{}{}
\patchcmd{\@boxed}{\boxed{#1#2}}{\colorbox{black!15}{$#1#2$}}{}{}
\makeatother

\title{\vspace{80mm}\lightgreyb CS203: Abstract Algebra \\
\lightgreyb Assignment $3$ Solutions}
\author{Ayush Bansal}
\date{\today}

\newtheorem{theorem}{Theorem}
\newtheorem{corollary}{Corollary}[theorem]
\newtheorem{conjecture}{Conjecture}
\newtheorem{lemma}{Lemma}
\newenvironment{solution}
  {\begin{proof}[Solution]}
  {\end{proof}}
\AfterEndEnvironment{theorem}{\noindent\ignorespaces}

\newenvironment{myenv}{\begin{adjustwidth}{1cm}{}}{\end{adjustwidth}}

\begin{document}
\setcounter{page}{4}
\setcounter{section}{1}
\setcounter{lemma}{3}
\section{Problem 2 Solution}{
  \setcounter{subsection}{2}
  \setcounter{equation}{5}
  \subsection{Part 3.}{
    Consider a prime number $p$ of the form $4n+1$, $p$ is prime in $\mathbb{Z}$.
    \begin{lemma}
      $p$ is not prime in $\mathbb{Z}[i]$ if $p$ is of the form $4n+1$.
    \end{lemma}
    \begin{proof}
    Since $p$ is of the form $4n+1$, by (2.5) we have some $x$ such that $(x^2+1)modp=0$, or we can say that $x^2+1=kp,k \in \mathbb{N}$. \newline
    Now assume $p$ is a prime number in $\mathbb{Z}[i]$. \newline
    We can write $x^2+1=(x+i)(x-i)$ and since $p \mid x^2+1$, by definition of prime in $\mathbb{Z}[i]$, $p \mid x+i$ or $p \mid x-i$. \newline
    \textbf{Case 1.} $p \mid x+i$
    \begin{myenv}
      Since $p \mid x+i$, $\frac{x}{p}+\frac{i}{p} \in \mathbb{Z}[i]$ but since we know that $i$ is a unit in $\mathbb{Z}[i]$, $\frac{i}{p} \notin \mathbb{Z}[i]$, thus our statement is a \textbf{Contradiction}.
    \end{myenv}
    \textbf{Case 2.} $p \mid x-i$
    \begin{myenv}
      Same argument as that of \textbf{Case 1}.
    \end{myenv}
    By above 2 cases, our lemma is proved.
    \end{proof}
    Now we will prove that there exists integers $x,y$ and $c$ such that $x^2+y^2=cp$ and $gcd(c,p)=1$.
    \begin{proof}
      Since we have proved that $p$ is not a prime in $\mathbb{Z}[i]$, we consider that it is product of 2 numbers from $\mathbb{Z}[i]$. \newline
      Let $p=(x+iy)(a+ib)$, doing computations on it and equating imaginary part to $0$, we get:
      \begin{align}
        p&=(ax-by)+i(ay+bx) \label{eq:6} \\
        ay+bx&=0 \\
        \frac{a}{b}&=-\frac{x}{y} \\
        kp&=x^2+y^2, k=\frac{x}{a}
      \end{align}
      Putting $k=1$ in the above equation, we get $p*1=x^2+y^2, x,y \in \mathbb{Z}$ and $gcd(1,p)=1$ and so the proof is complete
    \end{proof}
  }
  \subsection{Part 4.}{
    We are given that $p$ is prime number such that $x^2+y^2=cp$, where $x,y$ and $c$ are integers and $gcd(c,p)=1$. \newline
    I have to prove that $p$ is not a prime in $\mathbb{Z}[i]$. \newline
    \begin{proof}
      Assume that $p$ is a prime in $\mathbb{Z}[i]$. \newline
      Since $cp=x^2+y^2$, we can write it as $cp=(x+iy)(x-iy)$, since $p \mid (x+iy)(x-iy)$, by definition of prime in $\mathbb{Z}[i]$, either $p \mid (x+iy)$ or $p \mid (x-iy)$. \newline
      Considering either of the both cases we get that $\frac{x}{p},\frac{y}{p} \in \mathbb{Z}[i]$ and since $x,y \in \mathbb{Z}$, also units of $\mathbb{Z}[i]$ are $\pm1,\pm i$, and $p \mid x$ and $p \mid y$. \newline
      Using above result, $x=ap$ and $y=bp$, $a,b \in \mathbb{Z}$, putting these values in our original equation we get:
      \begin{align}
        cp=b^2p^2+a^2p^2 \label{eq:10} \\
        c=p(b^2+a^2)
      \end{align}
      Since equation (2.11) is a \textbf{Contradiction} to the fact that $gcd(c,p)=1$, thus $p$ is not a prime in $\mathbb{Z}[i]$.
    \end{proof}
  }
  \subsection{Part 5.}{
    We are given a prime number $p$ which is prime in $\mathbb{Z}$ but not in $\mathbb{Z}[i]$. \newline
    I have to prove that $p=a^2+b^2$ and $a,b \in \mathbb{Z}$.
    \begin{proof}
      Since $p$ is not a prime in $\mathbb{Z}[i]$, we can write $p$ as $p=(a+ib)(x+iy)$, where $a,b,x,y$ are integers. \newline
      Now doing the same computations done in (2.6),(2.7),(2.8) and (2.9), putting $k=1$, we get:
      \begin{align*}
        p=x^2+y^2, \quad x,y \in \mathbb{Z}
      \end{align*}
      The above equation is the one which I had to prove and so my proof is complete.
    \end{proof}
  }
  \subsection{Part 6.}{
    We are given that $p$ is a prime number in $\mathbb{Z}$ and is of the form $4n+1$. \newline
    I have to prove that $p=a^2+b^2$, where $a,b \in \mathbb{Z}$.
    \begin{proof}
      Using \textbf{Lemma 4}, we can say that since $p$ is of the form $4n+1$, $p$ will not be a prime in $\mathbb{Z}[i]$. \newline
      Since $p$ is not a prime in $\mathbb{Z}[i]$, from the proof of \textbf{2.5 Part 5.} we can say that $p=a^2+b^2$ for some integers $a$ and $b$, which concludes our proof.
    \end{proof}
  }
}
\newpage
\end{document}
