\documentclass{article}
\usepackage[a4paper]{geometry}
\usepackage[utf8]{inputenc}
\usepackage[sfdefault]{roboto}
\usepackage{graphicx}

\let\bold\textbf
\let\ital\textit
\let\smca\smallcaps

\begin{document}
\section*{Scatter Plots}
\begin{figure}[!htb]
    \centering
    \includegraphics{thread_1_scatter}
    \label{scatter1}
\end{figure}
This plot shows Execution time when executing using just \bold{one thread} and the time should increase as number of elements increases which is evident from the plot.

\newpage
\begin{figure}[!htb]
    \centering
    \includegraphics{thread_2_scatter}
    \label{scatter2}
\end{figure}
This plot shows Execution time when executing using \bold{two threads} and the time here is roughly half the value we got from executing on one thread.

\newpage
\begin{figure}[!htb]
    \centering
    \includegraphics{thread_4_scatter}
    \label{scatter4}
\end{figure}
This plot shows Execution time when executing using \bold{four threads} and the time here is roughly half the value we got from executing on two threads unless you have \bold{less than 4 cores} on your machine.

\newpage
\begin{figure}[!htb]
    \centering
    \includegraphics{thread_8_scatter}
    \label{scatter8}
\end{figure}
This plot shows Execution time when executing using \bold{eight threads} and the time here is halved one more time if we have \bold{greater than or equal to 8 cores} or it roughly remains the same.

\newpage
\begin{figure}[!htb]
    \centering
    \includegraphics{thread_16_scatter}
    \label{scatter16}
\end{figure}
This plot shows Execution time when executing using \bold{sixteen threads} and the time here is halved or roughly remains same.

\newpage
\section*{Line Plot}
\begin{figure}[!htb]
    \centering
    \includegraphics{thread_lineplot}
    \label{lineplot}
\end{figure}
This plot shows Average Execution for each thread count for certain number of elements. \newline
For low value of number of elements the time on increasing the thread count is unpredictable but usually increases. \newline
For high value of number of elements the time on increasing the thread count decreases till CPU core count becomes the bottleneck.

\newpage
\section*{Speedup Bar Plot}
\begin{figure}[!htb]
    \centering
    \includegraphics{thread_speedup}
    \label{speedup}
\end{figure}
This plot shows the average speedup values for different thread count for different number of elements. \newline
The values are averaged over the 100 samples for each thread count and number of elements.

\newpage
\section*{Error Bar Plot}
\begin{figure}[!htb]
    \centering
    \includegraphics{thread_errorbars}
    \label{errorbars}
\end{figure}
This plot shows the standard deviation from the mean (average) values for different thread count for different number of elements.
\end{document}

